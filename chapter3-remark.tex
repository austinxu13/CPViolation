\section{Introduction}
第三章讨论了非相对论量子力学中的\C \P \T 的问题,我的注记包括CPV一书和井樱书中的内容
\section{\C 变换}
电荷共轭变换在非相对论量子力学不容易给出,讨论电荷共轭变化,必须考虑电磁场

讨论含电磁场的拉式量
\begin{equation}
H = \frac{|\vec P|^2}{2m} - \frac e{2mc}[\vec A \cdot \vec P + \vec P \cdot \vec A] + \frac{e^2}{2mc^2}\vec A^2 + e\phi
\end{equation}
做下面的变换后,哈密顿算符是不变的:
\begin{equation}
    e\rightarrow-e, \vec A \rightarrow -\vec A, \phi\rightarrow -\phi
\end{equation}
同时,$\vec P$以及$\vec X$是不变的。
因此,我们应该将\C 定义为使得系统的$e,A,\phi$反号的变换。
显然有$\C^2 = 1$实际上,\C 变换的定义比较复杂,需要在有Dirac方程的时候做进一步的讨论。
\section{\P 变换}
在量子力学中讨论\P 变换,\P 变换将右手参考系变换为左手参考系。
将量子力学的态$\ket{\alpha}$转化为$\ket{\alpha}\rightarrow\P\ket{\alpha}$
位置算符的平均值应该有
\begin{equation}
    \bra{\alpha}\invUP\vec x \UP\ket{\alpha} = - \bra{\alpha}\vec x\ket{\alpha}
\end{equation}
因此应该有
\begin{equation}
    \UP x\invUP = -x
\end{equation}
\section{\T 变换}

